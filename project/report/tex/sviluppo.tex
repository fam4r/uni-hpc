\section{Sviluppo}

\subsection{Ghost area}

Entrambe le soluzioni proposte si avvalgono dell'utilizzo di una ghost area
inizializzata a zero (e mai più modificata).
Ciò permette di snellire notevolmente l'impatto sulle prestazioni
dell'operazione di propagazione, poiché, essendoci una regione di intorno
(\textit{halo}) al dominio, non sarà più necessario controllare che le celle
adiacenti a quella presa in esame siano ai bordi del dominio stesso.

\subsection{Accesso in memoria al dominio}

È stato inizialmente ipotizzato l'accesso alla matrice considerandola come un
array, in accordo con la rappresentazione in memoria delle matrici fornita dal
linguaggio C.
Ciò avrebbe permesso di scorrere in modo sequenziale la matrice stessa,
risparmiando le chiamate alla funzione \texttt{IDX}.

Ad esempio, l'operazione di incremento di energia sarebbe stata implementata nel
seguente modo:
\begin{minted}{c}
for (int i = 0; i < n; i++) {
    grid[i] += delta;
}
\end{minted}

Tuttavia, l'introduzione della ghost area atta a ridurre l'impatto di
prestazioni in altre aree della soluzione proposta ha reso preferibile (in
termini di gestione dell'accesso alla memoria) l'utilizzo della funzione
\texttt{IDX}, che contribuisce inoltre ad una buona comprensione e leggibilità
del codice sorgente.

\subsection{Implementazione OpenMP}

\subsubsection{Parallelizzazione della versione seriale}

L'inizializzazione della ghost area a zero non è stata parallelizzata poiché i
test effettuati hanno mostrato un degradamento, seppur minimo, delle
prestazioni. Si noti il confronto nella tabella di seguito:

\begin{table}[ht]
\begin{tabular}{rccc}
\cmidrule[\heavyrulewidth]{2-4}
 & Lato dominio & Numero passi & T\textsubscript{setup} (\textit{s})\\
 \cmidrule[\lightrulewidth]{2-4}
 seriale & \multirow{2}{*}{512} & \multirow{2}{*}{1000} & 0.00337177\\
 parallelizzato &&& 0.00454419\\
\cmidrule[\heavyrulewidth]{2-4}
\end{tabular}
\caption{caption}
\end{table}

Tale comportamento può essere dovuto ad un ulteriore carico di lavoro assegnato
allo scheduler per un'attività che mal si presta (soprattutto per l'accesso ai
lati sinistro e destro del dominio a causa dell'accesso in memoria) ad essere
parallelizzata.

Tutte le principali operazioni (incremento, conteggio, propagazione e calcolo
energia media) contengono dei costrutti \texttt{for} che sono stati
parallelizzati mediante la direttiva OpenMP \texttt{omp parallel for}.
In tale modo il carico di lavoro del ciclo a cui è applicata la direttiva viene
suddiviso tra i thread che concorrono all'esecuzione del programma, realizzando
la vera e propria parallelizzazione.

Nelle operazioni di conteggio e di calcolo dell'energia è stato inoltre
introdotto l'operatore di riduzione somma sulle rispettive variabili, in modo
tale da calcolare le somme in modo parallelo all'interno dei blocchi OpenMP\@.

\subsubsection{Altre ottimizzazioni}

\paragraph{collapse}

È stato valutato l'utilizzo dell'operatore OpenMP \texttt{collapse(2)} sui cicli
\texttt{for} innestati per collassare le due iterazioni in una, allargando lo
spazio di iterazione come da specifiche\cite{openmp2018reference}.

Tale direttiva ha tuttavia causato un degrado delle prestazioni (si veda la
tabella che segue) e pertanto non è stata implementata nella soluzione proposta.

\begin{table}[ht]
\begin{tabularx}{\linewidth}{rXXXXX}
\cmidrule[\heavyrulewidth]{2-6}
& Lato dom. & \# passi & $\overline{T}$\textsubscript{3-run}(\textit{s})
& cache-references (M/sec) & cache-misses (\%)*\\
\cmidrule[\lightrulewidth]{2-6}
senza \texttt{collapse(2)} & \multirow{2}{*}{256} & \multirow{2}{*}{100000} &
   13.33183 & 0.770 & 0.390\\
\cmidrule{4-6}
   con \texttt{collapse(2)} &&& 19.18336 & 0.560 & 0.402\\
\cmidrule[\heavyrulewidth]{2-6}
\end{tabularx}
\caption{caption}
\end{table}

*: of all cache refs.

Dai dati presenti in tabella è possibile osservare che, al contrario di come ci
si aspetterebbe, nonostante la direttiva \texttt{collapse} possa modificare i
\textit{chunk} di memoria a cui si accede, ciò non comporta un peggioramento
delle prestazioni nell'accesso alla memoria cache.

Di conseguenza si ipotizza che il degradamento delle prestazioni possa essere
legato ad una cattiva gestione da parte dello scheduler nell'organizzare il
lavoro dei thread.

\paragraph{scheduler}

È stato valutato anche l'utilizzo dell'operatore \texttt{scheduler}, testando il
parametro \texttt{type} nelle sue varianti \texttt{static}, \texttt{dynamic},
\texttt{guided}.
Si rimanda alle specifiche \cite{openmp2018reference} per ulteriori dettagli su
tale operatore.

Dopo aver condotto svariate prove si è riscontrato un aumento significativo
delle prestazioni impostando uno scheduling statico con dimensione dei blocchi
di 8.

La tabella \ref{tab:scheduler} mostra la differenza di tempistiche medie (su 3
esecuzioni) tra la versione in cui il parametro \texttt{scheduler} non è
specificato e quella in cui viene usato con parametri \texttt{static, 8}.

\begin{table}[ht]
\begin{tabularx}{\linewidth}{XXXX}
\toprule
Lato dom. & \# passi & $\overline{T}$\textsubscript{3-run}(\textit{s})&
$\overline{T}$\textsubscript{3-run}(\textit{s}) con \texttt{scheduler(static,8)}\\
\midrule
 256 & \multirow{3}{*}{100000} & 12.9581 & 10.3148 \\
 512 && 50.4409 & 40.5048 \\
 1024 && 250.4084 & 177.2195 \\
\bottomrule
\end{tabularx}
\caption{\label{tab:scheduler}Comparazione delle tempistiche di esecuzione
applicando uno scheduling statico.}
\end{table}

Tale direttiva ha quindi permesso una importante riduzione delle tempistiche,
compresa tra il 20 e il 30\% in meno rispetto alla precedente versione già
parallelizzata.

Altri tipi di scheduling o diverse dimensioni dei \textit{chunk} hanno
pareggiato o addirittura peggiorato le precedenti tempistiche.

%Poiché le quattro operazioni fondamentali necessitano di essere eseguite in uno
%specifico ordine, non è stato possibile valutare altre ottimizzazioni o API
%di OpenMP per ottimizzare l'esecuzione.
%La principale è \texttt{parallel for}.

\subsection{Implementazione CUDA}

Il dominio è stato partizionato, come da suggerimenti, in blocchi 2D organizzati
a loro volta in griglie 2D.

Per le operazioni effettuate nelle funzioni \texttt{count\_cells} e
\texttt{average\_energy} il dominio è stato invece partizionato in blocchi
monodimensionali, in accordo con la rappresentazione matriciale nel linguaggio
di programmazione C.

A tal proposito, per le sopracitate operazioni, è stata opportunamente
implementata l'operazione di riduzione mediante operatore somma.

In un primo momento, per semplicità, si è proceduto con la realizzazione della
soluzione senza l'utilizzo della memoria condivisa del Device.
