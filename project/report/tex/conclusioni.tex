\section{Conclusioni}

Tale relazione di progetto vuole portare all'attenzione tutte le fasi di
realizzazione del progetto stesso: l'analisi del modello, l'implementazione
delle versioni parallele e la valutazione delle prestazioni seguendo un
approccio quanto più metodico possibile.

Ogni sezione della relazione è inoltre accompagnata da considerazioni svolte
durante l'implementazione stessa, come, ad esempio, l'utilizzo di pattern,
clausole o l'adozione di una variante di una formula piuttosto che un'altra in
base a determinate condizioni.  Tali scelte sono state inoltre motivate con dati
raccolti durante le varie verifiche in corso d'opera.

\subsection{Note di sviluppo}

Sono stati realizzati diversi script che facilitano ed automatizzano
l'esecuzione su server remoti e la valutazione delle prestazioni.

Possono risultare di interesse i meccanismi ideati per trasferire, compilare,
eseguire l'algoritmo e scaricare i risultati da/verso server remoti utilizzando
il comando \texttt{scp} e l'opzione \texttt{ProxyJump} di SSH.

In ultimo sono stati realizzati degli script per l'automatizzazione del calcolo
dei parametri prestazionali quali tempistiche, speedup e scalabilità.

Si consultino i sorgenti per maggiori informazioni.
